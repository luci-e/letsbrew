%%%%%%%%%%%%%%%%%%%%%%%%%%%%%%%%%%%%%%%%%
% Beamer Presentation
% LaTeX Template
% Version 1.0 (10/11/12)
%
% This template has been downloaded from:
% http://www.LaTeXTemplates.com
%
% License:
% CC BY-NC-SA 3.0 \ttp://creativecommons.org/licenses/by-nc-sa/3.0/)
%
%%%%%%%%%%%%%%%%%%%%%%%%%%%%%%%%%%%%%%%%%

%----------------------------------------------------------------------------------------
%	PACKAGES AND THEMES
%----------------------------------------------------------------------------------------

\documentclass[12pt]{beamer}
\usepackage[absolute,overlay]{textpos}
% small logo down the page
%\logo{\makebox[1\paperwidth]{\includegraphics[width=.5cm,keepaspectratio]{images/sapienza-logo.png}}}

%\fontsize{12pt}{10}\selectfont


\usepackage{minted}
\usepackage{placeins}
\usepackage{graphicx}
\usepackage{tikz}
\usepackage{graphicx}
\usetikzlibrary{arrows,automata,positioning}
\usetikzlibrary{calc}
\usetikzlibrary{shapes}



\definecolor{color1}{HTML}{001000} % Color of the article title and sections
\definecolor{color2}{HTML}{822434} % Color of the boxes behind the abstract and headings
\definecolor{color4}{HTML}{006778} % Color of the boxes behind the abstract and headings
\definecolor{bla}{HTML}{000000} % Color of the boxes behind the abstract and headings
\definecolor{color3}{HTML}{202020} % Background color
    
%\setbeamertemplate{background canvas}{

%\begin{tikzpicture}
%\node[opacity=.1]{
%\includegraphics[width=2in , height=3in, keepaspectratio]%{images/sapienza-logo.png}};
%\end{tikzpicture}
%}
 % only for the image: http://ctan.org/pkg/mwe
%     \setbeamertemplate{background}{\includegraphics[width=\paperwidth]{images/sapienza-logo.png}}
%{\usebackgroundtemplate{
%       \begin{picture}
%            \includegraphics[width=\paperwidth]{images/sapienza-logo.png}
%       \end{picture}
%}%


% watermark logo , you can choose width and position in a pretty accurate way
%"anchor" and current page."x" play the game : north , south , etc.
\setbeamertemplate{background}{\tikz[overlay,remember picture, anchor=south]\node[opacity=.15]at ([yshift=1.5cm]current page.south){\includegraphics[width=3cm]{images/sapienza-logo.png}};}


\definecolor{color1}{HTML}{001000} % Color of the article title and subsections
\definecolor{color2}{HTML}{822434} % Color of the boxes behind the abstract and headings
\definecolor{color3}{HTML}{202020} % Background color

\definecolor{baseColor}{HTML}{822434} % Color of the boxes behind the abstract and headings
\mode<presentation> {

% The Beamer class comes with a number of default slide themes
% which change the colors and layouts of slides. Below this is a list
% of all the themes, uncomment each in turn to see what they look like.


%   \usetheme{Frankfurt}
%  \usetheme{Ilmenau}
%  \usetheme{JuanLesPins}
%  \usetheme{Luebeck}
%   \usetheme{PaloAlto}

\usetheme{Rochester}






% As well as themes, the Beamer class has a number of color themes
% for any slide theme. Uncomment each of these in turn to see how it
% changes the colors of your current slide theme.


% \usecolortheme{crane}
%\usecolortheme{orchid}
%   \usecolortheme{rose}
%  \usecolortheme{whale}


%\setbeamertemplate{footline} % To remove the footer line in all slides uncomment this line
\setbeamertemplate{footline}[page number] % To replace the footer line in all slides with a simple slide count uncomment this line

\setbeamertemplate{navigation symbols}{} % To remove the navigation symbols from the bottom of all slides uncomment this line

 \setbeamercolor*{palette primary}{use=structure,fg=color2,bg=color2!110 }
 \setbeamercolor*{palette secondary}{use=structure,fg=color3,bg=color3}
 \setbeamercolor{title}{fg=white,bg=baseColor}


}

\usepackage{tabularx}
\usepackage{pdfpages}
\usepackage{listings}
\usepackage{minted}
\usepackage{graphicx} % Allows including images
\usepackage{booktabs} % Allows the use of \toprule, \midrule and \bottomrule in tables
\usepackage{scrextend}
\changefontsizes{14pt}

\usepackage{tikz}
%----------------------------------------------------------------------------------------
%	TITLE PAGE
%----------------------------------------------------------------------------------------
\title[Short title]{IoT} % The short title appears at the bottom of every slide, the full title is only on the title page

\author{} % Your name
\institute[Sapienza] % Your institution as it will appear on the bottom of every slide, may be shorthand to save space
{
La Sapienza \\ % Your institution for the title page
\medskip
\textit{studenti.uniroma1.it} % Your email address

}
\date{\today} % Date, can be changed to a custom date

%%%%%%%%%%%%%%%%%%%%%%%%%%%%%%%%%%%%%%%%%%%%%%%%%
%%%   New style here

\mode<presentation>{\usetheme{AnnArbor}}
\usecolortheme{whale}

\setbeamercolor{frametitle}{parent=subsubsection in head/foot}
\setbeamercolor{frametitle right}{parent=subsection in head/foot}

\makeatletter
\pgfdeclarehorizontalshading[frametitle.bg,frametitle right.bg]{beamer@frametitleshade}{\paperheight}{%
    color(0pt)=(frametitle.bg);
    color(\paperwidth)=(frametitle right.bg)}

\AtBeginDocument{
    \pgfdeclareverticalshading{beamer@topshade}{\paperwidth}{%
        color(0pt)=(bg);
        color(4pt)=(black!50!bg)}
}

\addtobeamertemplate{headline}
{}
{%
    \vskip-0.2pt
    \pgfuseshading{beamer@topshade}
    \vskip-2pt
}


\setbeamertemplate{frametitle}
{%
    \nointerlineskip%
    \vskip-2pt%
    \hbox{\leavevmode
        \advance\beamer@leftmargin by -12bp%
        \advance\beamer@rightmargin by -12bp%
        \beamer@tempdim=\textwidth%
        \advance\beamer@tempdim by \beamer@leftmargin%
        \advance\beamer@tempdim by \beamer@rightmargin%
        \hskip-\Gm@lmargin\hbox{%
            \setbox\beamer@tempbox=\hbox{\begin{minipage}[b]{\paperwidth}%
                    \vbox{}\vskip-.75ex%
                    \leftskip0.3cm%
                    \rightskip0.3cm plus1fil\leavevmode
                    \insertframetitle%
                    \ifx\insertframesubtitle\@empty%
                    \strut\par%
                    \else
                    \par{\usebeamerfont*{framesubtitle}{\usebeamercolor[fg]{framesubtitle}\insertframesubtitle}\strut\par}%
                    \fi%
                    \nointerlineskip
                    \vbox{}%
                \end{minipage}}%
                \beamer@tempdim=\ht\beamer@tempbox%
                \advance\beamer@tempdim by 2pt%
                \begin{pgfpicture}{0pt}{0pt}{\paperwidth}{\beamer@tempdim}
                    \usebeamercolor{frametitle right}
                    \pgfpathrectangle{\pgfpointorigin}{\pgfpoint{\paperwidth}{\beamer@tempdim}}
                    \pgfusepath{clip}
                    \pgftext[left,base]{\pgfuseshading{beamer@frametitleshade}}
                \end{pgfpicture}
                \hskip-\paperwidth%
                \box\beamer@tempbox%
            }%
            \hskip-\Gm@rmargin%
        }%
        \nointerlineskip
        \vskip-0.2pt
        \hbox to\textwidth{\hskip-\Gm@lmargin\pgfuseshading{beamer@topshade}\hskip-\Gm@rmargin}
        \vskip-2pt
    }
\makeatother

\setbeamercolor{subsection in toc}{fg=red}
%%%

\setbeamercolor{structure}{fg=baseColor}%!80!black}
\setbeamercolor*{block title example}{fg=blue!50,bg= blue!10}
\setbeamercolor*{block body example}{fg= red,bg= blue!5}
\usefonttheme{structuresmallcapsserif}

\setbeamertemplate{footline}[page number]{} 

\setbeamercolor{headline}{fg=blue!90!black,bg=baseColor}
\setbeamercolor{palette primary}{fg=white,bg=baseColor}
\setbeamercolor{palette secondary}{fg=white,bg=baseColor}
\setbeamercolor{palette tertiary}{fg=white,bg=color4}
\setbeamercolor{frametitle}{fg=white,bg=baseColor}


\colorlet{titleleft}{baseColor}
\colorlet{titleright}{baseColor}

\setbeamercolor*{frametitle}{fg=white}

\makeatletter
\pgfdeclarehorizontalshading[titleleft,titleright]{beamer@frametitleshade}{\paperheight}{%
    color(0pt)=(titleleft);
    color(\paperwidth)=(titleright)}
\makeatother

%%%
%%%  End new style
%%%%%%%%%%%%%%%%%%%%%%%%%%%%%%%%%%%%%%%%%%%%%%%%




\begin{document}
\usemintedstyle{lovelace}







{ % all template changes are local to this group.
    \setbeamertemplate{navigation symbols}{}
    \begin{frame}[plain]
        \begin{tikzpicture}[remember picture,overlay]
            \node[at=(current page.center)] {
                \includegraphics[width=\paperwidth]{front.pdf}
            };
        \end{tikzpicture}
     \end{frame}
}


\section{Introduzione}




\subsection{}
\begin{frame}
    \frametitle{Introduzione}
Let's brew! - IoT project\\
IoT Tee/Coffee machine with remote control and data reporting.
\end{frame}
\section{System}
\begin{frame}
    \frametitle{System overview}
    
    \begin{itemize}
        \item STM32F4 board controls the boiler
        \item PC running python server as gateway
        \item ThingsBoard on remote server
    \end{itemize}
    Hardware:
    \begin{itemize}
        \item optocoupler
        \item triac to control the AC power
        \item 1000 Watts of power!
    \end{itemize}

\end{frame}
\subsection{sensors}
\begin{frame}
    \frametitle{Sensors}

    Voltage divider to measure temperature:
    \begin{itemize}
        \item NTC Thermistor 100k
        \item ADC to sample the divider voltage
        \item Resistor equations to find the NTC resistance
        \item Beta equation to find the temperature
    \end{itemize}

\end{frame}
\subsection{circuit}
\begin{frame}
    \frametitle{circuit}
    \includegraphics[width=\textwidth]{eagle.png}
\end{frame}
\subsection{Software}
\begin{frame}
    \frametitle{circuit}
Software features:
\begin{itemize}
    \item FreeRTOS
        \begin{itemize}
            \item threads
            \item timers
            \item mutexes
        \end{itemize}
    \item  HAL
        \begin{itemize}
            \item UART - DMA
            \item Bluetooth
            \item ADC
            \item GPIO
        \end{itemize}
\end{itemize}
\end{frame}

\section{Server}
\subsection{description}
\begin{frame}
    \frametitle{HTCPCP Server}
    The board implements a HTTP - like stateless server.
    The server can handle the following requests:
    \begin{itemize}
        \item Brew
        \item KeepWarm
        \item Status
        \item Abort
    \end{itemize}

\end{frame}
\subsection{server fsa}
\begin{frame}
    \frametitle{server fsa}
    
%\begin{left}
\begin{tikzpicture}[overlay, remember picture]
    \node[anchor=center] at (5,-2) {

\resizebox{1.2\textwidth}{!}{

\begin{tikzpicture}[>=latex',shorten >=1pt,node distance=5cm,on grid,auto]
    \node[state,initial, text width=2cm, align=center] (idle) {\begin{small}IDLE/\\Heater off\end{small}};
    \node[state, text width=2cm, align=center] (kwa) [above right=of idle] {\begin{tiny}Keep Warm Active/\\Heater on \end{tiny}};
    \node[state, text width=2cm, align=center] (kwi) [right=of kwa] {\begin{tiny}Keep Warm Idle/\\Heater off \end{tiny}};
  \node[state, text width=2cm, align=center] (brp) [above left=of idle] {\begin{tiny}Brew preheating/\\Heater on \end{tiny}};
  \node[state, text width=2cm, align=center] (brh) [above left=of brp] {\begin{tiny}Brewing/\\Heater on \end{tiny}};
  \node[state,accepting, text width=2cm, align=center] (err) [below right=of idle] {\begin{tiny}Error state/\\Heater off \end{tiny}};
  %\path[->] (q0-e) edge node {a} (q1-0);
  %\path[->] (q0-e) edge node {b} (q3-0);
\path[->] (idle) edge [bend left] node  {\tiny Brew, $T<50$ / OK} (brp);
  \path[->] (idle) edge [loop below] node {\tiny Brew, $T\geq50$ OR KW,$T<45$ / ERROR} (idle);
  \path[->] (brp) edge [loop right] node {\tiny Brew,KW / ERROR} (brp);
  \path[->] (brh) edge [loop right] node {\tiny Brew,KW / ERROR} (brh);
  \path[->] (kwa) edge [loop above] node {\tiny Brew,KW/ERR}(kwa);
  \path[->] (kwi) edge [loop above] node {\tiny Brew,KW/ERR}(kwi);

  %\path[->] (q0-1) edge [bend right] node {b} (q3-0);
  %\path[->] (q1-0) edge [bend left=90,looseness=2.5] node {a} (q3-1);
  \path[->] (brp) edge node {\tiny $T \geq 90$/Start T1,OK} (brh);
  \path[->] (idle) edge [bend left] node {\tiny KW,$T\geq 45$/OK Start T2} (kwa);
  \path[->] (kwi) edge[out = 140,in= 45,bend right ] node {\tiny $T<=45$/OK} (kwa);
  \path[->] (kwi) edge[out = -90, in = -45,bend left] node {\tiny T2 Expired/OK} (idle);
  \path[->] (kwa) edge node {\tiny $T>=60$/OK} (kwi);
  \path[->,red] (kwa) edge[out=-45,in=0,looseness=3] node[red] {\tiny $T<40$/ERROR} (err);
  \path[->] (kwa) edge[out=270,in=0,looseness=1] node[ ] {\tiny T2 Expired/OK} (idle);
  \path[->,red] (brh) edge [out=180,in=270,looseness=1] node [red] {\tiny $T < 80$/ERROR} (err);
  %\draw [->] (empt) ..  controls  ($(empt)-(2cm,8cm)$) and
  %  ($(empt)-(2cm,8cm)$) ..  (err);
  %\path[->] (q3-0) edge [bend right] node {b} (q0-1);
  %\path[->] (q3-1) edge [loop below] node {a} (q3-1);
  %\path[->] (q3-1) edge node {b} (q0-1);
\end{tikzpicture}
}

};
\end{tikzpicture}
%\end{left}

\end{frame}

\subsection{Board code}
\begin{frame}
    \frametitle{Board code}

    \begin{itemize}
        \item Controller Class - C++
        \item Request string parsing - C++
        \item FreeRTOS interface - C
        \item FreeRTOS thread functions - C
        \item Hardware configuartion - C
    \end{itemize}
\end{frame}

\section{Gateway}
\subsection{Python server}

\begin{frame}
    \frametitle{Python server}

The python server enables both bluetooth functionality and acts as serial gateway to push information to ThingsBoard.
\begin{itemize}
    \item Uses MQTT
    \item Polls via serial using Status command
\end{itemize}

\end{frame}


\subsection{ThingsBoard}
\begin{frame}
    \frametitle{ThingsBoard}
    \begin{itemize}
        \item Hosted on premises on remote server
        \item Gathers data via MQTT
        \item Displays information:
        \item    \begin{itemize}
                \item Temperature
                \item Usage statistics
                \item Time to operation completion
                \item Power consumption statistics
            \end{itemize}

   \end{itemize}
\end{frame}

\section{Conclusione}


\begin{frame}
    \frametitle{The end.}
    \begin{center}

        \begin{Huge}        Thank you! \end{Huge}
\end{center}
    \end{frame}










%----------------------------------------------------------------------------------------

\end{document} 
